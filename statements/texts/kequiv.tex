\begin{problem}{K-equivalence}{kequiv.in}{kequiv.out}{2 seconds}

% Author: Mikhail Dvorkin, mikhail.dvorkin@gmail.com

Consider a set $K$ of positive integers.

Let $p$ and $q$ be two non-zero decimal digits. Call them $K$-equivalent
if the following condition applies:

\begin{quote}
For every $n \in K$, if you replace one digit $p$ with $q$ or one digit $q$ with
$p$ in the decimal notation of $n$ then the resulting number will be
an element of $K$.
\end{quote}

For example, when $K$ is the set of integers divisible by $3$, the digits
$1$, $4$, and $7$ are $K$-equivalent. Indeed, replacing a $1$ with a $4$ in
the decimal notation of a number never changes its divisibility by $3$.

It can be seen that $K$-equivalence is an equivalence relation
(it is reflexive, symmetric and transitive).

You are given a finite set $K$ in form of a union of disjoint finite intervals
of positive integers.

Your task is to find the equivalence classes of digits 1 to 9.

\InputFile

The first line contains $n$, the number of intervals composing the set $K$
($1 \le n \le 10\,000$).

Each of the next $n$ lines contains two positive integers $a_i$ and $b_i$ that
describe the interval $[a_i, b_i]$
(i. e. the set of positive integers between $a_i$ and $b_i$, inclusive), where
$1 \le a_i \le b_i \le 10^{18}$. Also, for $i \in [2..n]$: $a_i \ge b_{i-1} + 2$.

\OutputFile

Represent each equivalence class as a concatenation of its elements,
in ascending order.

Output all the equivalence classes of digits 1 to 9, one at a line, sorted
lexicographically.

\Example

\begin{example}
	\exmp{
1
1 566
}{
1234
5
6
789
}%
\exmp{
1
30 75
}{
12
345
6
7
89
}%
\end{example}

\end{problem}
