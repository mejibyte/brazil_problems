\begin{problem}{Garbling Game}{garbling.in}{garbling.out}{2 seconds}

% Author: Pavel Kuznetsov, Text: Roman Elizarov

Pavel had invented a new game with a matrix of integer numbers. He takes $r \times c$ matrix with $r$ rows
and $c$ columns that is filled with numbers from $1$ to $rc$ left to right and top to bottom
($1$ is written in the upper-left corner, $rc$ is written in the lower-right corner).
Then he starts to rearrange the numbers is the matrix by following the rules that are explained below and
writes down a sequence of numbers on a separate piece of paper. He calls it \emph{garbling} of the matrix.

The rules of rearrangement are defined by \emph{garbling map} that is $(r-1) \times (c-1)$ matrix
of letters L, R, and N. Initial $4 \times 5$ matrix and the sample $3 \times 4$ garbling map for it
are shown below.

\newcommand{\rr}[1]{\multicolumn{1}{|>{\columncolor[gray]{.8}}c|}{#1}}
\newcommand{\ww}[1]{(#1)}

\begin{center}
\begin{tabular}{p{5cm}p{5cm}}
\begin{tabular}{|c|c|c|c|c|}
\hline
\rr{\ww{1}} & \rr{2} & 3 & 4 & 5 \\
\hline
\rr{6} & \rr{7} & 8 & 9 & 10 \\
\hline
11 & 12 & 13 & 14 & 15 \\
\hline
16 & 17 & 18 & 19 & 20 \\
\hline
\end{tabular}
&
\begin{tabular}{|c|c|c|c|}
\hline
L & R & L & R \\
\hline
N & L & L & R \\
\hline
L & N & N & L \\
\hline
\end{tabular}
\\
\end{tabular}
\end{center}

Pavel garbles the matrix in a series of turns. On his first turn Pavel takes the number in the first row and the
first column (it is put in parenthesis on the above picture for clarity) and writes it down.

Having written down the number he performs one \emph{garbling turn}:

Pavel looks at the letter in the garbling map that corresponds to the position of the number he had
just written down (one the first turn it is the letter in the upper-left corner). Depending on
the letter in the garbling map the $2 \times 2$ block of the matrix whose upper-left corner
contains the number he had just written (highlighted in the above picture) is rearranged in the
following way:

\begin{itemize}
\item{R} --- the block is rotated \emph{clockwise}.
\item{L} --- the block is rotated \emph{counterclockwise}.
\item{N} --- Pavel does not change the matrix on this turn.
\end{itemize}

On the second turn Pavel takes the number in the first row and second column, writes it down, and performs
the garbling turn, and so on. In $c-1$ turns he finishes the first row and moves to the second row and so on
he proceeds left to right and top to bottom. In $(r-1)(c-1)$ turns he had written down $(r-1)(c-1)$ numbers and
garbled the whole matrix, so he starts again in the upper-left corner continuing garbling the matrix from left
to right and top to bottom.

The matrices below show the effect of the first four turns with the sample garbling map.

\begin{tabular}{@{}p{4.5cm}@{}p{4.5cm}@{}p{4.5cm}@{}p{4.5cm}@{}}
\begin{tabular}{|c|c|c|c|c|}
\hline
2 & \rr{\ww{7}} & \rr{3} & 4 & 5 \\
\hline
1 & \rr{6} & \rr{8} & 9 & 10 \\
\hline
11 & 12 & 13 & 14 & 15 \\
\hline
16 & 17 & 18 & 19 & 20 \\
\hline
\end{tabular}
&
\begin{tabular}{|c|c|c|c|c|}
\hline
2 & 6 & \rr{\ww{7}} & \rr{4} & 5 \\
\hline
1 & 8 & \rr{3} & \rr{9} & 10 \\
\hline
11 & 12 & 13 & 14 & 15 \\
\hline
16 & 17 & 18 & 19 & 20 \\
\hline
\end{tabular}
&
\begin{tabular}{|c|c|c|c|c|}
\hline
2 & 6 & 4 & \rr{\ww{9}} & \rr{5} \\
\hline
1 & 8 & 7 & \rr{3} & \rr{10} \\
\hline
11 & 12 & 13 & 14 & 15 \\
\hline
16 & 17 & 18 & 19 & 20 \\
\hline
\end{tabular}
&
\begin{tabular}{|c|c|c|c|c|}
\hline
2 & 6 & 4 & 3 & 9 \\
\hline
\rr{\ww{1}} & \rr{8} & 7 & 10 & 5 \\
\hline
\rr{11} & \rr{12} & 13 & 14 & 15 \\
\hline
16 & 17 & 18 & 19 & 20 \\
\hline
\end{tabular}
\\
\end{tabular}

The following sequence of numbers is written down in the first 4 turns: \texttt{1 7 7 9}.
On 5th turn the number from the second row and the first column is written, but the matrix remains unchanged,
since the second row and the first column of the garbling map contains N. In six turns Pavel
gets \texttt{1 7 7 9 1 8}.

Given the garbling map and the number of moves Pavel makes in this game, find out how
many times each number gets written down by Pavel. You need to provide the
answer modulo $10^5$.

\InputFile

The first line of the input file contains three integer numbers --- $r$, $c$, and $n$, where
$r$, $c$ ($2 \le r$, $c \le 300$) are the dimensions of the initial matrix,
$n$ ($0 \le n < 10^{100}$) is the number of turns Pavel makes.

The following $r-1$ lines contain garbling map with $c-1$ characters R, L, or N on a line.

\OutputFile

Write to the output file $rc$ lines with one integer number per line. On $i$-th line write
the number of times number $i$ gets written down by Pavel modulo $10^5$ while he makes his $n$ turns.

\Example

\begin{example}
\exmp{
4 5 6
LRLR
NLLR
LNNL
}{
2
0
0
0
0
0
2
1
1
0
0
0
0
0
0
0
0
0
0
0
}%
\exmp{
4 5 666666
LRLR
NLLR
LNNL
}{
37038
37038
0
0
30864
37036
11112
30864
30864
30864
30864
30864
11110
30865
18519
30864
30864
0
18518
18518
}%
\end{example}

\end{problem}
